%%%%%%%%%%%%%%%%%
% This is an sample CV template created using altacv.cls
% (v1.1.4, 27 July 2018) written by LianTze Lim (liantze@gmail.com). Now compiles with pdfLaTeX, XeLaTeX and LuaLaTeX.
% 
%% It may be distributed and/or modified under the
%% conditions of the LaTeX Project Public License, either version 1.3
%% of this license or (at your option) any later version.
%% The latest version of this license is in
%%    http://www.latex-project.org/lppl.txt
%% and version 1.3 or later is part of all distributions of LaTeX
%% version 2003/12/01 or later.
%%%%%%%%%%%%%%%%

%% If you need to pass whatever options to xcolor
\PassOptionsToPackage{dvipsnames}{xcolor}

%% If you are using \orcid or academicons
%% icons, make sure you have the academicons 
%% option here, and compile with XeLaTeX
%% or LuaLaTeX.
% \documentclass[10pt,a4paper,academicons]{altacv}

%% Use the "normalphoto" option if you want a normal photo instead of cropped to a circle
% \documentclass[10pt,a4paper,normalphoto]{altacv}

\documentclass[10pt,a4paper]{altacv}
%% AltaCV uses the fontawesome and academicon fonts
%% and packages. 
%% See texdoc.net/pkg/fontawecome and http://texdoc.net/pkg/academicons for full list of symbols.
%% 
%% Compile with LuaLaTeX for best results. If you
%% want to use XeLaTeX, you may need to install
%% Academicons.ttf in your operating system's font 
%% folder.


% Change the page layout if you need to
\geometry{left=1cm,right=9cm,marginparwidth=6.8cm,marginparsep=1.2cm,top=1.25cm,bottom=1.25cm,footskip=2\baselineskip}

% Change the font if you want to.

% If using pdflatex:
\usepackage[T1]{fontenc}
\usepackage[utf8]{inputenc}
\usepackage[default]{lato}

% If using xelatex or lualatex:
% \setmainfont{Lato}

% Change the colours if you want to
\definecolor{Mulberry}{HTML}{72243D}
\definecolor{SlateGrey}{HTML}{2E2E2E}
\definecolor{LightGrey}{HTML}{666666}
\definecolor{MyBlue}{HTML}{3388aa}
\definecolor{MyRed}{HTML}{aa3333}
\colorlet{heading}{MyBlue}
\colorlet{accent}{MyRed}
\colorlet{emphasis}{SlateGrey}
\colorlet{body}{LightGrey}

% Change the bullets for itemize and rating marker
% for \cvskill if you want to
\renewcommand{\itemmarker}{{\small\textbullet}}
\renewcommand{\ratingmarker}{\faCircle}
%% sample.bib contains your publications
\addbibresource{sample.bib}

\usepackage[colorlinks]{hyperref}

\begin{document}

\name{Evan Sitt}
\tagline{Computer Science Student}
\photo{2.8cm}{2020-01-14_CV_2v1}
\personalinfo{%
  % Not all of these are required!
  % You can add your own with \printinfo{symbol}{detail}
  \email{Sitt.Evan@gmail.com }
  \phone{+1 312 270 0648}
  \mailaddress{1011 E Kevin Circle, Palatine, IL 60074 United States}
  \location{Budapest, Hungary}
%   \homepage{http://Yeet.Codes/}
%   \twitter{@twitterhandle}
  \linkedin{linkedin.com/in/evan-sitt/}
  \github{github.com/ParadoxChains}
  %% You MUST add the academicons option to \documentclass, then compile with LuaLaTeX or XeLaTeX, if you want to use \orcid or other academicons commands.
%   \orcid{orcid.org/0000-0000-0000-0000}
}

%% Make the header extend all the way to the right, if you want. 
\begin{fullwidth}
\makecvheader
\end{fullwidth}

%% Depending on your tastes, you may want to make fonts of itemize environments slightly smaller
% \AtBeginEnvironment{itemize}{\small}


%% Provide the file name containing the sidebar contents as an optional parameter to \cvsection.
%% You can always just use \marginpar{...} if you do
%% not need to align the top of the contents to any
%% \cvsection title in the "main" bar.
\cvsection[page1sidebar]{Experience}

\cvevent{Instructor (Functional Programming)}{E\"otv\"os Lor\'and University}{September 2019 -- Ongoing}{Budapest, Hungary}
\begin{itemize}
\item Introduce incoming first year students to the functional programming paradigm, from good coding habits to basic algorithms, by using practical coding methodology.
\item Organize and manage curriculum and consultations to promote better student progression and performance.
\item Recruited and organized a team of 12 undergraduate students in furthering their pursuit of functional programming with development of an digital signal processing framework.  
\end{itemize}

\divider

\cvevent{Student Developer}{Ericsson Hungary}{March 2019 -- December 2019}{Budapest, Hungary}
\begin{itemize}
\item Have proper knowledge and skill in coding with Erlang for telecommunication applications.

\item Write functional tests for new functionality developed by the team.

\item Address customer raised Trouble Reports and Issues in a timely manner via debugging and testing.

\item Extend and refactor legacy code for better performance, efficiency, and maintainability.

\end{itemize}

\cvsection{Projects}

\cvevent{Music Generation and Sound Processing in a  Functional Programming Paradigm}{E\"otv\"os Lor\'and University}{2019-2020 Academic Year}{}
\small{Digital synthesis is a cross discipline application used in fields such as music, telecommunication,
and others. The nature of digital synthesis involving multiple tracks as well as parallel postprocesses
lends itself naturally to the functional programming paradigm. The paper demonstrates this by
creating a fully functional, cross platform, standalone synthesizer application framework implemented
in a pure lazy functional language. The application handles MIDI input and produces wav output played
by any multimedia player. Therefore, it can serve as a preprocessor for users who intend to create digital
signals before transcribing them into a digital or physical media.}

% \begin{itemize}
% \item Details
% \end{itemize}

\divider

\cvevent{Increased Interest in Studying Functional Programming via Integrated Application}{E\"otv\"os Lor\'and University}{2018-2019 Academic Year}{}
\small{Functional programming represents a modern tool for applying and implementing software. The state
of the art in functional programming reports an increasing number of methodologies in this paradigm.
However, extensive interdisciplinary applications are missing. Our goal is to increase student interest
in pursuing further studies in functional programming with the use of an application: the ray tracer.
We conducted a teaching experience, with positive results and student feedback, described here in
this paper.}

\medskip

% \cvsection{A Day of My Life}

% % Adapted from @Jake's answer from http://tex.stackexchange.com/a/82729/226
% % \wheelchart{outer radius}{inner radius}{
% % comma-separated list of value/text width/color/detail}
% \wheelchart{1.5cm}{0.5cm}{%
%   6/8em/accent!30/{Sleep,\\beautiful sleep}, 
%   4/8em/accent!8/{Watching news},
%   3/7em/accent!8/Meeting \& Talking with friends,
%   8/8em/accent!55/Day time Job,
%   2/10em/accent!10/Sports and relaxation,
%   5/6em/accent!20/Spending time with family
% }

\clearpage
% \cvsection[page2sidebar]{Publications}

% \nocite{*}

% \printbibliography[heading=pubtype,title={\printinfo{\faBook}{Books}},type=book]

% \divider

% \printbibliography[heading=pubtype,title={\printinfo{\faFileTextO}{Journal Articles}},type=article]

% \divider

% \printbibliography[heading=pubtype,title={\printinfo{\faGroup}{Conference Proceedings}},type=inproceedings]

%% If the NEXT page doesn't start with a \cvsection but you'd
%% still like to add a sidebar, then use this command on THIS
%% page to add it. The optional argument lets you pull up the 
%% sidebar a bit so that it looks aligned with the top of the
%% main column.
% \addnextpagesidebar[-1ex]{page3sidebar}

\end{document}
